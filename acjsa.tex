%% This is file `elsarticle-template-2-harv.tex',
%%
%% Copyright 2009 Elsevier Ltd
%%
%% This file is part of the 'Elsarticle Bundle'.
%% ---------------------------------------------
%%
%% It may be distributed under the conditions of the LaTeX Project Public
%% License, either version 1.2 of this license or (at your option) any
%% later version.  The latest version of this license is in
%%    http://www.latex-project.org/lppl.txt
%% and version 1.2 or later is part of all distributions of LaTeX
%% version 1999/12/01 or later.
%%
%% The list of all files belonging to the 'Elsarticle Bundle' is
%% given in the file `manifest.txt'.
%%
%% Template article for Elsevier's document class `elsarticle'
%% with harvard style bibliographic references
%%
%% $Id: elsarticle-template-2-harv.tex 155 2009-10-08 05:35:05Z rishi $
%% $URL: http://lenova.river-valley.com/svn/elsbst/trunk/elsarticle-template-2-harv.tex $
%%

%%\documentclass[preprint,authoryear,12pt]{elsarticle}

%% Use the option review to obtain double line spacing
%% \documentclass[authoryear,preprint,review,12pt]{elsarticle}

%% Use the options 1p,twocolumn; 3p; 3p,twocolumn; 5p; or 5p,twocolumn
%% for a journal layout:

%% Astronomy & Computing uses 5p
%% \documentclass[final,authoryear,5p,times]{elsarticle}
\documentclass[final,authoryear,5p,times,twocolumn]{elsarticle}

%% if you use PostScript figures in your article
%% use the graphics package for simple commands
%% \usepackage{graphics}
%% or use the graphicx package for more complicated commands
\usepackage{graphicx}
%% or use the epsfig package if you prefer to use the old commands
%% \usepackage{epsfig}

%% The amssymb package provides various useful mathematical symbols
\usepackage{amssymb}
%% The amsthm package provides extended theorem environments
%% \usepackage{amsthm}

\usepackage[pdftex,pdfpagemode={UseOutlines},bookmarks,bookmarksopen,colorlinks,linkcolor={blue},citecolor={green},urlcolor={red}]{hyperref}
\usepackage{hypernat}

%% Alternatives to hyperref for testing
%\usepackage{url}
%\newcommand{\htmladdnormallinkfoot}[2]{#1\footnote{\texttt{#2}}}
%\newcommand{\htmladdnormallink}[1]{\texttt{#1}}
%\newcommand{\href}[2]{\texttt{#2}}

%% The lineno packages adds line numbers. Start line numbering with
%% \begin{linenumbers}, end it with \end{linenumbers}. Or switch it on
%% for the whole article with \linenumbers after \end{frontmatter}.
%% \usepackage{lineno}

%% natbib.sty is loaded by default. However, natbib options can be
%% provided with \biboptions{...} command. Following options are
%% valid:

%%   round  -  round parentheses are used (default)
%%   square -  square brackets are used   [option]
%%   curly  -  curly braces are used      {option}
%%   angle  -  angle brackets are used    <option>
%%   semicolon  -  multiple citations separated by semi-colon (default)
%%   colon  - same as semicolon, an earlier confusion
%%   comma  -  separated by comma
%%   authoryear - selects author-year citations (default)
%%   numbers-  selects numerical citations
%%   super  -  numerical citations as superscripts
%%   sort   -  sorts multiple citations according to order in ref. list
%%   sort&compress   -  like sort, but also compresses numerical citations
%%   compress - compresses without sorting
%%   longnamesfirst  -  makes first citation full author list
%%
%% \biboptions{longnamesfirst,comma}

% \biboptions{}

\journal{Astronomy \& Computing}

%% Make single quotes look right in verbatim mode
\usepackage{upquote}

\usepackage{upgreek}

\usepackage{color}

% Aim to be consistent, and correct, about how we refer to sections
\newcommand*\secref[1]{Sect.~\ref{#1}}
\newcommand*\appref[1]{\ref{#1}}

\begin{document}

\begin{frontmatter}

%% Title, authors and addresses

%% use the tnoteref command within \title for footnotes;
%% use the tnotetext command for the associated footnote;
%% use the fnref command within \author or \address for footnotes;
%% use the fntext command for the associated footnote;
%% use the corref command within \author for corresponding author footnotes;
%% use the cortext command for the associated footnote;
%% use the ead command for the email address,
%% and the form \ead[url] for the home page:
%%
%% \title{Title\tnoteref{label1}}
%% \tnotetext[label1]{}
%% \author{Name\corref{cor1}\fnref{label2}}
%% \ead{email address}
%% \ead[url]{home page}
%% \fntext[label2]{}
%% \cortext[cor1]{}
%% \address{Address\fnref{label3}}
%% \fntext[label3]{}

\title{Lessons learned from an observatory/data center partnership:
  The JCMT Science Archive}

%% use optional labels to link authors explicitly to addresses:
%% \author[label1,label2]{<author name>}
%% \address[label1]{<address>}
%% \address[label2]{<address>}

\author[noao]{Frossie Economou\corref{cor1}}
\ead{frossie@noao.edu}
\author[cadc]{S\'{e}verin Gaudet}
\author[cornell,jac]{Tim Jenness}
\author[jac]{Russell O.\ Redman}
\author[cadc]{Sharon Goliath}
\author[cadc]{Patrick Dowler}
\author[cadc]{David Schade}
\author[uherts,jac]{Antonio Chrysostomou}


\cortext[cor1]{Corresponding author}

\address[noao]{National Optical Astronomy Observatory, 950 N Cherry Ave, Tucson, AZ 85719, USA}
\address[cornell]{Department of Astronomy, Cornell University, Ithaca,
  NY 14853, USA}
\address[jac]{Joint Astronomy Centre, 660 N.\ A`oh\=ok\=u Place, Hilo, HI
  96720, USA}
\address[cadc]{Canadian Astronomy Data Centre, National Research Council of Canada, 5071 West Saanich Road., Victoria, BC V9E 2E7, Canada}
\address[uherts]{Centre for Astrophysics Research, University of Hertfordshire, College Lane, Hatfield, Hertfordshire AL10 9AB, UK}

\begin{abstract}
%% Text of abstract

{\color{red}
Abstract here
}

\end{abstract}

\begin{keyword}
%% keywords here, in the form: keyword \sep keyword

%% MSC codes here, in the form: \MSC code \sep code
%% or \MSC[2008] code \sep code (2000 is the default)

JCMT, VO, Archives

\end{keyword}

\end{frontmatter}

% \linenumbers

%% Journal abbreviations
\newcommand{\mnras}{MNRAS}
\newcommand{\aap}{A\&A}
\newcommand{\aaps}{A\&AS}
\newcommand{\pasp}{PASP}
\newcommand{\apj}{ApJ}
\newcommand{\apjs}{ApJS}
\newcommand{\qjras}{QJRAS}
\newcommand{\an}{Astron.\ Nach.}
\newcommand{\ijimw}{Int.\ J.\ Infrared \& Millimeter Waves}
\newcommand{\procspie}{Proc.\ SPIE}
\newcommand{\aspconf}{ASP Conf. Ser.}

%% Applications


%% Links
\newcommand{\ascl}[1]{\href{http://www.ascl.net/#1}{ascl:#1}}

%% main text

\section{Origins}

{\color{red}
In (19xx/2000? oh God I forget) -

The JCMT Science Archive \citep{2008SPIE.7016E..16G,2008ASPC..394..450E,2008ASPC..394..135G,2011ASPC..442..203E} \ldots

There was a SCUBA-2 mapping workshop in Vancouver in April 2003. This is likely the
one where Severin was approached for the first time. NPR requested that
David and Severin work on proposal for VO integration of JCMT archive
be presented at November 2003 JCMT Board meeting.

SCUBA-2 pipeline PDR held at Victoria in Sep 2003.

We were in Vancouver and Victoria in March 2005.

Approved by JCMT Board in May 2005 \citep{2005JCMTN23}. CADC got
CA\$2.5M/yr for 3 years of operations which was announced in
June. PPARC approved 2 FTE of ex-Starlink effort to support JSA
pipeline. JAC supplied 2 FTE for internal development of processing
infrastructure.

JCMT Data Users' Group (JDUG) created early 2006. \citep{2006JCMTN24R}

JLS approved in July 2005 \citep{2005JCMTN23}. Leiden meeting was Jan 2005.

}

\section{Motivation: Observatory}

Submillimeter data has traditionally been rather esoteric, closer to
radio than the optical/infrared regime familiar to most
astronomers. Raw data is typically in time series format, and requires
in-house algorithms for transformation to science-ready formats such
as spectra or images. Calibration is difficult due to the dominant and
highly variable effect of the water vapour in Earth's atmosphere
\citep[e.g.,][]{2002MNRAS.336....1A,2013MNRAS.430.2534D}.

{\color{red}
[visual example here - scary raw data]
}

JCMT invested significant effort in automated data reduction based on
the ORAC-DR pipeline framework
\citep[][\ascl{1310.001}]{1999ASPC..172...11E,1999ASPC..172..171J,2005ASPC..347..585G,2008ASPC..394..565J}. In
many cases these automatically generated products were publication
quality, and thanks to a constantly updated calibration model, better
than what an inexperienced astronomer could be expected to achieve on
their own. Moreover with the advent of large bolometer arrays such as
SCUBA-2 \citep{2013MNRAS.430.2513H}, this data could be processed in
maps that resulted in image data that could be readily understood by
non-submm specialists.

{\color{red}
[visual example here - pretty scuba2 map]
}

The JCMT had in-house experience with setting up a data archive in the
``filing cabinet'' sense of allowing users to search and retrieve data.

Indeed, distribution of publication-quality data became an issue of
the highest priority with the advent of the JCMT Legacy Survey
Programme \citep{2010HiA....15..797C,2008ASPC..394..450E} using the
SCUBA-2 and HARP/ACSIS \citep{2009MNRAS.399.1026B} instruments. Aside
from the normal desire to provide a uniformly reduced product to the
survey teams, the processing demands for this data required a
non-trivial IT infrastructure . Due to the iterative algorithms
utilized by the SCUBA-2 map-maker
\citep[SMURF;][]{2013MNRAS.430.2545C}, the more data could be
co-processed, the higher the fidelity of the solution, and circa 2008,
machines with specifications such as 64\,GB of RAM were not readily
available to the typical JCMT observer. So there were intrinsic
reasons to have a JCMT Science Archive allowing the survey consortia
to download the processed products. But ultimately, usage of such an
archive would be dominated by JCMT users retrieving their own data, or
after the proprietary period elapsed, other JCMT users working in the
same scientific areas.

JCMT formed a strong interest in going further, and exposing its
high-value data product to data-mining astronomers who would not have
a priori knowledge either of JCMT in particular or sub-mm astronomy in
general. To that end, the VO data discovery and publication protocols
seemed like a natural choice for reaching the large parts of the
astronomical community that were oblivious to its existence. VO
publication would also have the advantage of exposing the JCMT data
sets to workhorse tools that VO-savvy astronomers already used, such
as TOPCAT \citep[][\ascl{1101.010}]{2005ASPC..347...29T} and Aladin
\citep[][\ascl{1112.019}]{2005ASPC..347..193O}.

However. despite being well convinced of the desirability of
leveraging the VO tools and services for JCMT data, the observatory
had the usual constraints of time and effort. The small Scientific
Computing Group was busy with supporting the entire non-hardware-controlling
software suite at both JCMT and UKIRT (operated by the same
organisation) as well as developing data reduction for new instruments
and supporting the JCMT Legacy Surveys. The ability to develop a
VO-aware data center and support the demands of the hoped-for
increased usage base was just not there.

What JCMT had, however, was a pre-existing collaboration with CADC,
which hosted the older JCMT data archive \citep{1997ASPC..125..397T} for the benefit of the
Canadian astronomical community, Canada being one of the three
international partners funding the JCMT (the other two being the
United Kingdom and the Netherlands). CADC had early involvement in VO
protocols, was a productive developer and enthusiastic supporter of VO
standards, and was known to ``eat its own dog food'' by using many of
these interfaces and services internally.

\section{Motivation: Data Centre}

{\color{red}
(bribes accepted for not entitling this CADC: Knights in Shining
Armour) (though *GOD* that would wind up Bob)

(Double check with CADC folks. My recollection here was:
}
CADC already had an varied collection of data from a variety of
telescopes and space missions \citep{1994ASPC...61..123C,2008SPIE.7016E..16G}. Keen to be able to extend its holdings
with a small amount of effort, CADC developed CAOM \citep[the Common
Archive Object Model;][]{2008ASPC..394..426D}, an extensive and
versatile data model that allowed any data to be ingested in the CADC
archive provided it could be adequately mapped to it.

One of the main attractions of the JCMT data set was its significant
departure from many of the common forms of other astronomical data,
that predominantly came from optical and IR instrumentation. JCMT data
frequently comes in Galactic co-ordinates, up to 4 dimensions (2
spatial, wavelength and polarization) and has a diverse number of
products, from spectra, to image previews, to data cubes, maps and
clump catalogues. CADC was planning the next version of their data
model, CAOM, and realized that JCMT provided an excellent stretch to
their model; if JCMT data could be described in CAOM, CADC would be
in the unprecedented position of being able to accept almost any data
set from future observatories with minimal changes to their system.

{\color{red}
(visual: CAOM example?)
}

Another advantage in working with JCMT on its datasets, was the high
level of completion and accuracy that JCMT provided in its
metadata. Astronomical data archiving with poor or incomplete metadata
becomes problematic verging onto untenable, especially when trying to
meet the bar of a ``Science Archive'', when science-ready data products
have to be delivered to the user. JCMT's dedication to high-fidelity
metadata and quick response in the rare case of problems made this an
attractive test data set.

{\color{red}
(Making this up - needs sanity check)
}

Meanwhile, CADC was working on the Canadian Advanced Network for
Astronomical Research \citep[CANFAR;][]{2010SPIE.7740E..51G} project aiming to
support a cloud-like model for astronomical data reduction. The system
is based on giving the user a VM with the appropriate software,
environment and data access. The user then defines a number of jobs
that are serviced on a Condor ({\color{red} is it?}) compute platform.

This service is clearly of great utility to the Canadian astronomical
community dealing with large data volumes, but it was also ideally
suited for servicing JCMT's need for data processing. In fact, CADC's
work with CANFAR enabled JCMT to adopt a relatively unusual continuous
data release model (see below).


\section{CANFAR and A Continuous Release Model}

Using CADC's CANFAR pipeline infrastructure and the capabilities of
JCMT/UKIRT's ORAC-DR automated data reduction, the JCMT Science
Archive adopted a model of continuous release
\citep{2011ASPC..442..203E}. As data was taken it was pushed for
reduction and was ingested at CADC in the same 24-hour period it was
observed. Thus, high-quality science products were published in the VO
as soon as the PI had access to them. Moreover, with every major
improvement in the data reduction software, data could be re-processed
and again immediately released.

Continuous release made the VO publication mechanisms even more useful
than they are in the normal data discovery process, as product
availability is, from the point of view of the astronomer,
unpredictable rather coming in fixed, scheduled, announced ``data
releases''.

\section{The optical-centric VO}

In the early days of the Virtual Observatory, the focus was
specifically on simple protocols \citep{siap,cone} to replace pre-existing
web services for image retrieval and cone search; with retrieval of
individual spectra coming somewhat later in VO developments
\citep{ssap,splatvo}. These were the pressing issues of the optical
community and this discussion dominated early protocol development.

Data cubes were seen as a task for the future as it was
felt that they were products that were not yet in the mainstream and
optical/IR IFU instruments \citep[such as
UIST;][]{2004SPIE.5492.1160R} were seen as something interesting for
the future. This was frustrating given that JCMT heterodyne
observations regularly generated cubes and with the arrival of ACSIS
in 2006, gigabyte data cubes were commonplace. There was no standard
available for making all these cubes available to the VO and it is
only recently \citep[e.g.,][]{2014AAS...22325505T} that a cube access
protocol has been approached with any seriousness, driven mainly, in
the USA, by ALMA and JWST developments \citep[e.g.,
MIRI;][]{2010SPIE.7731E..10W}.

Another peculiarity of sub-millimeter data is the lack of point
sources. Most Galactic objects are extended and dust and gas from
large clouds, outflows and filamentary structures are missed by
standard source extraction algorithms such as SExtractor
\citep[][\ascl{1010.064}]{1996A&AS..117..393B}. Instead, algorithms
such as Fellwalker \citep[][\ascl{1311.007}]{2007ASPC..376..425B} and
Clumpfind \citep[][\ascl{1107.014}]{1994ApJ...428..693W} which detect
source emission in irregularly-shaped clumps. VO ConeSearch was not
set up for this eventuality and the best we could hope for was to
provide a catalogue that indicated the peak of the emission. To work
around this problem clump catalogues are generated with the clump
outline approximated by a polygon specified in STC-S format
\citep{2010ASPC..434..213B}. These outlines can then be retrieved
using TAP for analysis or plotting. This is certainly less convenient
for the end user than a clump equivalent of ConeSearch so we are
extending the facilities in GAIA
\citep[][\ascl{1403.024}]{2009ASPC..411..575D} to hide the TAP
interface. Ideally a variant of ConeSearch will be developed that
worked for extended irregular sources.

\section{Metrics}

{\color{red}
Usage stats? Do we have any usage stats specifically for the VO services?
}

\section{Current Status}

{\color{red}
Extended collaboration to include UKIRT cassegrain data
\citep{adassxxiii_P01}.

\citep{2014JCMTN35..19J}

Generation of ``all-sky'' HEALpix products of all public HARP/ACSIS
and SCUBA-2 data \citep{2014SPIE9152-93,2014JCMTN35..20B}.

Clump and peak catalogues created with outlines specified in STC-S
\citep{2014JCMTN35..21G} for easy retrieval
over TAP.

Migrated to CAOM-2.0 \citep{2013ASPC..475..159R,2012ASPC..461..339D}
with lots of new features for VO and Advanced Search.
}

\section{Lessons Learned}

The JCMT Science Archive collaboration was a high successful foray
into VO publication via an observatory-data center collaboration.

Elements that we believe led to this success:

\begin{itemize}
\item VO publication was a common goal with significant organizational
  buy-in for both parties from the start, and was a primary technical
  goal of the collaboration rather than an afterthought.

\item Within that shared vision, there was a clear division of
  expertise and responsibilities for each side, allowing each
  organisation to focus on its proximate technical goals. Both
  organisations had ``skin in the game'' that were served by the
  technical work undertaken, which allowed this work to be carried out
  without any kind of external funding (each institution supported its
  own share of the work out of its normal budgetary process).

\item Each organisation worked from a position of strength based on an
  advanced, robust and mature software architecture, allowing
  development to focus on new functionality and interfaces between the
  two systems. This minimized the communication overheads commonly
  associated with distributed project.

\item There was a high level of pre-existing trust between the two
  groups from their previous relationship leading to minimal need for
  contractual language or management oversight. Indeed the entire
  collaboration's only official document was a two paragraph
  memorandum of understanding.
\end{itemize}

\section{Recommendations}

In the general case, for observatories that do not understand the
mechanisms or benefits of VO publication, collaboration with a
motivated VO-involved data center that has the appropriate
infrastructure and keeps up to date with the IVOA standards process is
a far more effective choice than trying to develop those capabilities
in-house, especially since there seems to be confusion in the
observatory community as to what ``VO publication'' involves and what
are the merits of doing it.

However, in order to be able to properly leverage the capability of a
modern multi-mission data center, a fanatical devotion to correct and
complete metadata should be considered a pre-requisite.

\section*{Acknowledgments}

The James Clerk Maxwell Telescope is operated by the Joint Astronomy
Centre on behalf of the Science and Technology Facilities Council of
the United Kingdom, the National Research Council of Canada, and
(until 31 March 2013) the Netherlands Organisation for Scientific
Research.  The Canadian Astronomy Data Centre is operated by the
National Research Council of Canada with the support of the Canadian
Space Agency.

%% References
%%
%% Following citation commands can be used in the body text:
%%
%%  \citet{key}  ==>>  Jones et al. (1990)
%%  \citep{key}  ==>>  (Jones et al., 1990)
%%
%% Multiple citations as normal:
%% \citep{key1,key2}         ==>> (Jones et al., 1990; Smith, 1989)
%%                            or  (Jones et al., 1990, 1991)
%%                            or  (Jones et al., 1990a,b)
%% \cite{key} is the equivalent of \citet{key} in author-year mode
%%
%% Full author lists may be forced with \citet* or \citep*, e.g.
%%   \citep*{key}            ==>> (Jones, Baker, and Williams, 1990)
%%
%% Optional notes as:
%%   \citep[chap. 2]{key}    ==>> (Jones et al., 1990, chap. 2)
%%   \citep[e.g.,][]{key}    ==>> (e.g., Jones et al., 1990)
%%   \citep[see][pg. 34]{key}==>> (see Jones et al., 1990, pg. 34)
%%  (Note: in standard LaTeX, only one note is allowed, after the ref.
%%   Here, one note is like the standard, two make pre- and post-notes.)
%%
%%   \citealt{key}          ==>> Jones et al. 1990
%%   \citealt*{key}         ==>> Jones, Baker, and Williams 1990
%%   \citealp{key}          ==>> Jones et al., 1990
%%   \citealp*{key}         ==>> Jones, Baker, and Williams, 1990
%%
%% Additional citation possibilities
%%   \citeauthor{key}       ==>> Jones et al.
%%   \citeauthor*{key}      ==>> Jones, Baker, and Williams
%%   \citeyear{key}         ==>> 1990
%%   \citeyearpar{key}      ==>> (1990)
%%   \citetext{priv. comm.} ==>> (priv. comm.)
%%   \citenum{key}          ==>> 11 [non-superscripted]
%% Note: full author lists depends on whether the bib style supports them;
%%       if not, the abbreviated list is printed even when full requested.
%%
%% For names like della Robbia at the start of a sentence, use
%%   \Citet{dRob98}         ==>> Della Robbia (1998)
%%   \Citep{dRob98}         ==>> (Della Robbia, 1998)
%%   \Citeauthor{dRob98}    ==>> Della Robbia


%% References with bibTeX database:

\bibliographystyle{model2-names-astronomy}
\bibliography{acjsa}

%% Authors are advised to submit their bibtex database files. They are
%% requested to list a bibtex style file in the manuscript if they do
%% not want to use model2-names.bst.

%% References without bibTeX database:

% \begin{thebibliography}{00}

%% \bibitem must have one of the following forms:
%%   \bibitem[Jones et al.(1990)]{key}...
%%   \bibitem[Jones et al.(1990)Jones, Baker, and Williams]{key}...
%%   \bibitem[Jones et al., 1990]{key}...
%%   \bibitem[\protect\citeauthoryear{Jones, Baker, and Williams}{Jones
%%       et al.}{1990}]{key}...
%%   \bibitem[\protect\citeauthoryear{Jones et al.}{1990}]{key}...
%%   \bibitem[\protect\astroncite{Jones et al.}{1990}]{key}...
%%   \bibitem[\protect\citename{Jones et al., }1990]{key}...
%%   \harvarditem[Jones et al.]{Jones, Baker, and Williams}{1990}{key}...
%%

% \bibitem[ ()]{}

% \end{thebibliography}

\end{document}

%%
%% End of file `elsarticle-template-2-harv.tex'.
